% By Tiger Sachse
% This is my first Latex document ever, so forgive the inevitable bad practices.

\documentclass{article}

\usepackage[margin=1.0in]{geometry}
\usepackage[utf8]{inputenc}
\usepackage{xcolor}
\usepackage{listings}
\usepackage[english]{babel}
\usepackage[hidelinks]{hyperref}

\setlength{\parskip}{1em}

\hypersetup{
    colorlinks,
    urlcolor=blue
}

\begin{document}

\begin{center}
{\Huge
    PL/0 User Manual
}

{\Large 
    Written by Tiger Sachse

    2018
}
\end{center}

\pagebreak
Here be a table of contents

- what is pl/0?

- how to setup/compile the program

- modes/flags

- Language specification

- language explanation

- Example programs

\pagebreak

\section*{What is PL/0?}
PL/0 is a simple, educational programming language meant to be used to teach compiler
design. It was invented in 1976 by Niklaus Wirth and it stands for
\textit{Programming Language Zero}. It has a limited number of features which make
it an impractical language to write real programs in, however its small size helps
allow student-designed compilers to remain small and \textit{relatively} simple.

\section*{What is Plum?}
Plum is an interpreter designed to run PL/0. It is modal, and is capable of parsing,
scanning, compiling, assembling, and executing PL/0 in sequence or separately. This means
that Plum is capable not only of compiling source into bytecode, but also of executing
that bytecode on an included, stack-based virtual machine. A description of each of Plum's
modes, as well as additional options, is presented later in this document.

Plum is derived from PLVM, an acronym for PL/0 Virtual Machine.
The name is no longer fully descriptive, as this program has evolved from a virtual machine into a
full-fledged interpreter, however it's cool, so it stays!

\section*{Plum Installation}
Plum is designed to be easy to install and use! The program can be found 
\href{https://www.github.com/tgsachse/plum}{at this link}. Plum can be downloaded from
the repository manually, or it can be cloned using \emph{git} like so:
\begin{lstlisting}[language=bash]
    $ git clone https://www.github.com/tgsachse/plum.git
\end{lstlisting}
To build, use the provided build script like this:
\begin{lstlisting}[language=bash]
    $ ./build.sh
\end{lstlisting}
This will produce an executable program named \emph{plum} in the same folder as
the build script. This executable is the entire interpreter and can be used to run
your PL/0 files! See the next couple sections for an explanation of the interpreter's
modes and options.

\pagebreak

\section*{Modes of Operation}

Plum is a modal program, meaning it has several modes of operation. What follows
is a list of Plum's modes, as well as brief descriptions of what each mode does.
\begin{itemize}
    \item run

        This mode takes a PL/0 source program as input, scans it, parses it, and then
        executes it on the virtual machine. This is the primary command of the interpreter
        that "does it all."

    \item scan

        This mode takes a PL/0 source program as input and produces a list of lexemes
        (tokenizations of the source code) as output. These lexemes can be translated
        into executable bytecode in the parse mode.

    \item parse
        
        This mode takes a list of PL/0 lexemes as input and produces executable bytecode
        for the virtual machine. This bytecode is "machine language," a sequence of numbers
        that the virtual machine understands as instructions.

    \item compile

        This mode takes a PL/0 source program as input and produces executable bytecode
        as output. It is functionally the same as passing a source program into the scan
        mode, and then parsing the results.

    \item execute

        This mode takes PL/0 bytecode as input and executes that bytecode on the virtual
        machine.
\end{itemize}



\end{document}
